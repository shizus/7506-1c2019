\documentclass{article}
\usepackage[utf8]{inputenc}

\title{Análisis Exploratorio de Datos \\ Trabajo Práctico 1 - Organización de Datos}
\author{Nombre de grupo}
\date{Poner fecha}






\usepackage{natbib}
\usepackage{graphicx}
\newcommand\tab[1][1cm]{\hspace*{#1}}
\usepackage{placeins}

\begin{document}
\begin{figure}
    \centering
    \makebox[\textwidth]{\includegraphics[width=250pt]{logofiuba.jpg}}
\end{figure}

\maketitle

\FloatBarrier
\begin{center}
        \begin{tabular}{ |c|c|c| }
          \hline
          Nombre & Padrón & Mail \\
          \hline\hline
          Álvarez, Federico & 99266 & fede.alvarez1997@gmail.com \\
          \hline
          La Torre, Gabriel & 87796 & latorregab@gmail.com \\
          \hline
          Medrano, Lucas Nicolás & 99247 & lucasmedrano97@gmail.com \\
          \hline
          Piro Martino, Ariel & 99469 & ariel.piro@hotmail.com \\
          \hline
        \end{tabular}
\end{center}
\FloatBarrier

\newpage

\tableofcontents
\newpage
\section{Introduction}
	\tab En el trabajo se hace un análisis exploratorio de un set de datos provistos por la empresa Jampp. En el mismo se encuentra información de subastas, instalaciones, clicks, entre otros.\\
	\tab Primero se hara una visión general de los archivos (installs.csv, clicks.csv, auctions.csv, events.csv) para entender la distribución y la cantidad de datos, el significado de las columnas, reconocer las columnas que no aportan información (por ejemplo, las que tienen todos sus valores nulos), reconocer el tipo de datos en cada columna, seguido de un análisis mas profundo para obtener mas información de los datos. Luego se hará un análisis global, buscando información relativa a los archivos en conjunto, permitiendo obtener otro tipo de información.

\section{Análisis individual de archivos}

\subsection{auctions.csv}

\subsection{clicks.csv}

\subsection{events.csv}

\subsection{installs.csv}

\section{Análisis de archivos en conjunto}

\section{Conclusion}


\bibliographystyle{plain}
\bibliography{references}
\end{document}
